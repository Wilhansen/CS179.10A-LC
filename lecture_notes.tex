%!TEX TS-program = xelatex
%!TEX encoding = UTF-8 Unicode

\documentclass[12pt]{report}
\usepackage{geometry}                % See geometry.pdf to learn the layout options. There are lots.
\geometry{a4paper}                   
%\usepackage[parfill]{parskip}    % Activate to begin paragraphs with an empty line rather than an indent
\usepackage{graphicx}
\usepackage{amssymb, amsmath, url,gensymb}
\usepackage[compact]{titlesec}
\usepackage{fontspec,xltxtra,xunicode}

\usepackage{tikz}
\usepackage{tkz-euclide}
\usetikzlibrary{angles,quotes}
\usetkzobj{all}
\usetikzlibrary{shapes, calc, decorations, intersections}

\DeclareMathOperator{\atan}{atan2}
\DeclareMathOperator{\acos}{acos}

\title{CS179.10A Lecture Notes}
\author{Wilhansen Li}

\begin{document}
\maketitle
\tableofcontents

\chapter{Vector Math}
\section{Trigonometry Review}
\subsection{Radians}
\begin{align*}
\text{toRad}(\theta) &= \theta \frac{\pi}{180\degree} \\
\text{toDeg}(\psi) &= \psi\frac{180\degree}{\pi}
\end{align*}
To compute the arc length of a circle with radius $r$ at angle $\theta$:
$$
2\pi r \left(\frac{\theta}{2\pi}\right) = r\theta
$$
give that $\theta$ is in radians.
\subsection{Trigonometric Functions}
\begin{center}
\begin{tikzpicture}
	[scale=3,line cap=round,
	% Styles
	axes/.style=,
	important line/.style={very thick},
	information text/.style={rounded corners,fill=red!10,inner sep=1ex}]
% Colors
	\colorlet{anglecolor}{green!50!black}
	\colorlet{sincolor}{red}
	\colorlet{tancolor}{orange!80!black}
	\colorlet{coscolor}{blue}
% The graphic
	\draw[help lines,step=0.5cm] (-1.4,-1.4) grid (1.4,1.4);
	\draw (0,0) circle [radius=1cm];
	\begin{scope}[axes]
		\draw[->] (-1.5,0) -- (1.5,0) node[right] {$x$} coordinate(x axis);
		\draw[->] (0,-1.5) -- (0,1.5) node[above] {$y$} coordinate(y axis);
		\foreach \x/\xtext in {-1, -.5/-\frac{1}{2}, 1}
			\draw[xshift=\x cm] (0pt,1pt) -- (0pt,-1pt) node[below,fill=white] {$\xtext$};
		\foreach \y/\ytext in {-1, -.5/-\frac{1}{2}, .5/\frac{1}{2}, 1}
			\draw[yshift=\y cm] (1pt,0pt) -- (-1pt,0pt) node[left,fill=white] {$\ytext$};
	\end{scope}
	\filldraw[fill=green!20,draw=anglecolor] (0,0) -- (3mm,0pt) arc [start angle=0, end angle=30, radius=3mm];
	\draw (15:2mm) node[anglecolor] {$\alpha$};
        \draw[important line,sincolor]
		(30:1cm) -- node[left=1pt,fill=white] {$\sin \alpha$} (30:1cm |- x axis); \draw[important line,coscolor]
		(30:1cm |- x axis) -- node[below=2pt,fill=white] {$\cos \alpha$} (0,0);
        \path [name path=upward line] (1,0) -- (1,1);
        \path [name path=sloped line] (0,0) -- (30:1.5cm);
        \draw [name intersections={of=upward line and sloped line, by=t}]
        [very thick,orange] (1,0) -- node [right=1pt,fill=white] {$\displaystyle \tan \alpha \color{black}=
        \frac{{\color{red}\sin \alpha}}{\color{blue}\cos \alpha}$} (t);
        \draw (0,0) -- (t);
%        \draw[xshift=1.85cm]
%        node[right,text width=6cm,information text] {
%        The {\color{anglecolor} angle $\alpha$} is $30^\circ$ in the
%        example ($\pi/6$ in radians). The {\color{sincolor}sine of
%        $\alpha$}, which is the height of the red line, is
%        \[
%        {\color{sincolor} \sin \alpha} = 1/2. \]
%        By the Theorem of Pythagoras ...
%        };
\end{tikzpicture}
\end{center}

\subsection{Trigonometric Identities}

We have the following basic identities:
\begin{align*}
\tan(\theta) = \frac{\sin(\theta)}{\cos(\theta)} = \frac{y}{x} = slope \\
\sin^2(\theta) + \cos^2(\theta) = 1
\end{align*}
Angle addition identities:
\begin{align*}
\sin(\alpha + \beta) = \sin(\alpha)\cos (\beta) + \cos(\alpha)\sin(\beta) \\
\cos(\alpha + \beta) = \cos(\alpha)\cos (\beta) - \sin(\alpha)\sin(\beta)
\end{align*}
Double angle identities:
\begin{align*}
\sin(2\alpha) &= 2\sin(\alpha)\cos (\alpha) \\
\cos(2\alpha) &= \cos^2(\alpha)- \sin^2(\alpha)
\end{align*}
Half angle identity:
\begin{align*}
\cos\left(\frac{a}{2}\right) = \pm \sqrt{\frac{1 + \cos{a}}{2}}
\sin\left(\frac{a}{2}\right) = \pm \sqrt{\frac{1 - \cos{a}}{2}}
\end{align*}
Law of cosines:
\begin{tikzpicture}
    \tkzDefPoint(5,2){A}
    \tkzDefPoint(0,0){B}
    \tkzDefPoint(6.8,0){C}
    \tkzDefPoint(5,0){E}

    % \tkzLabelPoints[above,font=\sansmath\sffamily](A)
    % \tkzLabelPoints[below left,font=\sansmath\sffamily](B)
    % \tkzLabelPoints[below right,font=\sansmath\sffamily](C)

    % Draw polygon
    \tkzDrawPolygon[fill=gray!10](A,B,C)

    % Label line segments
    \tkzLabelSegment[below](B,C){$a$}
    \tkzLabelSegment[right](A,C){$b$}
    \tkzLabelSegment[above left](A,B){$c$}

    % Mark angles
    \tkzMarkAngle[arc=l,size=0.6cm,fill=green!30](A,C,B)
    \tkzLabelAngle[pos=0.42](A,C,B){$C$}

    % Draw polygon
    \tkzDrawPolygon(A,B,C)
\end{tikzpicture}

\begin{align*}
c^2 = a^2 + b^2 -2ab\cos(C)
\end{align*}
\subsection{Polar Coordinates $\leftrightarrow$ Cartesian Coordinates}
\begin{align*}
\lbrack r, \theta \rbrack &= \langle r\cos\theta, r\sin\theta \rangle \\
\langle x, y \rangle &= \lbrack \sqrt{x^2 + y^2}, \atan(y, x) \rbrack
\end{align*}

\subsubsection{Cylindrical coordinates}
TODO
\subsubsection{Spherical Coordinates}
TODO
\section{Vector Math}
Let $u = \langle a_1, a_2, \dots, a_n \rangle$, $v = \langle b_1,b_2, \dots, b_n \rangle$, $w = \langle c_1, c_2, \dots, c_n \rangle$, and $s \in \mathbb{R}$
\subsection{Basic Operations}
\begin{align*}
u + v &= \langle a_1 + b_1, a_2 + b_2, \dots, a_n + b_n \rangle \\
su &= \langle sa_1, sa_2, \dots, sa_n \rangle \\
|u| &= \sqrt{a_1^2 + a_2^2 + \dots + a_n^2}
\end{align*}

Note that $|u| >= 0$ and $|u| = 0 \Leftrightarrow u = \vec{0}$

\subsection{Perp Operator}
For 2d vectors $u = \langle x, y \rangle$:
$u^\perp = \langle -y, x \rangle$. This is also know as "turn left" and derived from rotating the vector by $\frac{\pi}{2}$:

$$
\begin{bmatrix}
\cos\frac{\pi}{2} & -\sin\frac{\pi}{2}\\
\sin\frac{\pi}{2} & \cos\frac{\pi}{2}\\
\end{bmatrix}\begin{bmatrix}
x\\
y
\end{bmatrix} = 
\begin{bmatrix}
0 & -1\\
1 & 0\\
\end{bmatrix}\begin{bmatrix}
x\\
y
\end{bmatrix} =
\begin{bmatrix}
-y\\
x
\end{bmatrix}
$$

\subsection{Normalization}
Normalization is the process of making a vector have unit length but still point to the same direction, the normalized vector is denoted by a hat symbol on top of it:
$$
\hat{u} = \frac{u}{|u|}
$$

As convention, $\hat{x} = \langle 1, 0, 0 \rangle$, $\hat{y} = \langle 0, 1, 0 \rangle$, and $\hat{z} = \langle 0, 0, 1 \rangle$.

\subsection{Dot Product}
$$
u \cdot v = a_1b_1 + a_2b_2 + \dots + a_nb_n = \sum_{i=1}^{n}a_ib_i
$$
Properties of the dot product:
\begin{align*}
u \cdot v &= v \cdot u\\
(su) \cdot v &= s(u \cdot v) = u\cdot (sv) \\
u^2 &= u \cdot u = |u|^2
\end{align*}
$u \neq \vec{0}$ and $v \neq \vec{0} $ are orthogonal (perpendicular) if and only if $u \cdot v = 0$

$$
(u - v)^2 = (u - v)\cdot(u - v) = u^2 -2u\cdot v + v^2 = u^2 + v^2 -2u\cdot v
$$
Which leads to:
$$
u \cdot v = |u||v|\cos\theta
$$
where $\theta$ is the angle between $u$ and $v$.

$$
u \cdot v = \begin{cases}
> 0 \Leftrightarrow \theta < \frac{\pi}{2},\\
= 0 \Leftrightarrow \theta = \frac{\pi}{2}, \\
< 0 \Leftrightarrow \theta > \frac{\pi}{2}
\end{cases}
$$

\subsection{Cross Product}
For 2D vectors:
$$
u \times v = a_1b_2 - a_2b_1 = \begin{vmatrix}a_1 & a_2 \\ b_1 & b_2 \end{vmatrix}
$$
For 3D vectors:
$$
u \times v = \langle a_2b_3 - a_3b_2, -(a_1b_3 - a_3b_1), a_1b_2 - a_2b_1\rangle = \begin{vmatrix}\hat{x} & \hat{y} & \hat{z} \\ a_1 & a_2 & a_3 \\ b_1 & b_2 & b_3\end{vmatrix}
$$

Note that you can get a similar result for 2D using the 3D definition by setting the third coordinate to be $0$, that is:
$$
u \times v =  \begin{vmatrix}\hat{x} & \hat{y} & \hat{z} \\ a_1 & a_2 & 0 \\ b_1 & b_2 & 0\end{vmatrix} = \langle 0, 0, a_1b_2 - a_2b_1\rangle
$$

Properties of the cross product:
\begin{align*}
u \times v &= -(v \times u)\\
u \times v &= |u||v|\sin\theta\\
(su) \times v &= s(u \times v) = u \times (sv)\\
(u \times v) \cdot u &= (u \times v) \cdot v = 0 \\
(u \times v) \times w &\neq u \times (v \times w) \\
(u + v)\times w &= (u \times w) + (v \times w)
\end{align*}
\subsection{Triple Product}
\subsubsection{Scalar Triple Product}
$$(u \times v) \cdot w = u \cdot (v \times w)$$
\subsubsection{Vector Triple Product}
$$u \times (v \times w) = v(u \cdot w) - w(u \cdot v)$$
\subsection{Applications (1)}
	\subsubsection{Angle Between Vectors}
	$$
	\theta = \acos\left(\frac{u \cdot v}{|u||v|}\right)
	$$
	Note $\theta \geq 0$ by virtue of 1) the $\acos$ function whose range is $[0, \pi]$ and 2) $\acos$'s inability to distinguish points on $+y$ from points on $-y$. Remember, $\cos$ maps the points of a unit circle to its x-position. As a result, this does not distinguish the direction of the angle. It does not say whether $v$ is located clockwise or counter-clocwise to $u$. To get such information, we can use the cross product:
	$\sin$ doesn't give us the full angle since it does not distinguish angles $0, \frac{\pi}{2}$ to angles $\frac{\pi}{2}, \pi$
	
	\subsubsection{Triangle Area and Orientation}
	
	To check if 3 points, ${p_0, p_1, p_2}$ are collinear, simply check of the triangle formed by the 3 points is degenerate, that is:
	$$
	(p_1-p_0)\times (p_2-p_0) = 0
	$$
	
	To check if a triangle ${p_0, p_1, p_2}$ is obtuse, acute, or right.
	$$
	min((p_i - p_j) \cdot (p_k - p_j))_{ (i,j,k) \in \{0,1,2\}^3}
	\begin{cases}
            > 0 &\text{acute},\\
            = 0 & \text{right}, \\
            < 0 & \text{obtuse}
        \end{cases}
	$$
	
	\subsubsection{Volume of a Parallelepiped}
	TODO
\subsection{Projection}
\subsection{Reflection}
\subsection{Applications (2)}
	\subsubsection{Line Orientation}
\section{Exercises}

\chapter{Lines and Polygons}
\chapter{Bounding Regions}
\chapter{Spatial Data Structures}
\chapter{Voronoi Regions and Delunay Triangulation}
\end{document}  